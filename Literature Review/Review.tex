\documentclass[11pt, a4paper]{article}
\usepackage[english]{babel}
\usepackage{ragged2e}
\usepackage[utf8x]{inputenc}
\usepackage{amsmath}
\usepackage{graphicx}
\usepackage[colorinlistoftodos]{todonotes}
\usepackage[font=small,labelfont=bf]{caption}
\usepackage[top=2.0in, bottom=1.25in, left=1.20in, right=1.20in]{geometry}

\title{\vspace{-7cm}}
\author{}
\date{}

\begin{document}
\maketitle

\section{Literature Review / Background}

\subsection{Game Theory}
Game theory is the study of behaviours and mathematical models which result from the decisions and strategies of two or more rational players in either cooperative or non-cooperative strategy games. Applications of game theory have manifested in social science, psychology, mathematics and many more fields of study; however, the root interactions lie in strategic games such as the prisoners dilemma or tit-for-tat. Game theory was introduced and popularised by mathematician John von Neumann, who first proved an optimal strategy for zero-sum games with perfect information such as chess or go called the minimax theorem in 1928. This theorem indicates that in such games, there is a pair of strategies for each player which allows them to minimise their maximum losses, while considering all possible response moves of the opponent. \\
\noindent
After von Neumann published his initial paper on game theory, he published a book entitled, "Theory of Games and Economic Behaviour" - this became a milestone for game theory as it established a foundation for becoming a unique, academic and economic field. Within this book, von Neumann fixates mainly on non-cooperative games and/or zero-sum games - within this was a method of identifying consistent solutions/strategies for both players in two-person zero-sum games.

\noindent

\clearpage

\end{document}