\documentclass[11pt, a4paper]{article}
\usepackage[english]{babel}
\usepackage{ragged2e}
\usepackage[utf8x]{inputenc}
\usepackage{amsmath}
\usepackage{graphicx}
\usepackage[colorinlistoftodos]{todonotes}
\usepackage[font=small,labelfont=bf]{caption}
\usepackage[top=2.0in, bottom=1.25in, left=1.20in, right=1.20in]{geometry}

\title{\vspace{-7cm}}
\author{}
\date{}

\begin{document}
\maketitle

\section{Literature Review / Background}

\subsection{Game Theory}
Game theory is the study of behaviours and mathematical models which result from the decisions and strategies of two or more economically rational players in either cooperative or non-cooperative strategy games. Applications of game theory have manifested in social science, psychology, mathematics and many more fields of study; however, the root interactions lie in strategic games such as the prisoner's dilemma or tit-for-tat. Game theory was introduced and popularised by mathematician John von Neumann, who first proved an optimal strategy for zero-sum games with perfect information such as chess or go called the minimax theorem in 1928. This theorem indicates that in such games, there is a pair of strategies for each player which allows them to minimise their maximum losses, while considering all responsive moves of the opponent. \\
\noindent
After von Neumann published his initial paper on game theory, he published a book co-authored by economist Oskar Morgenstein entitled, "Theory of Games and Economic Behaviour". Within this book, von Neumann fixates mainly on non-cooperative games and/or zero-sum games; but most importantly, identified a method of finding consistent solutions and strategies for both players in two-person zero-sum games. This work became a milestone for game theory as it established a foundation for becoming a unique discipline. \\
\noindent
Following this, numerous advancements in game theory occurred during the 1950s - mathematicians Merill Flood and Melvin Dresher experimented mathematical and game versions of the prisoner's dilemma for the American think tank corporation, RAND (Research and Development). In the same year, John Forbes Nash Jr published his dissertation on non-cooperative games which contained the first definitions of the Nash equilibrium - an important milestone for adaptive strategy in game theory. He proved that in every n-player non-zero sum game, a Nash equilibrium existed, assuming the game had a finite number of actions. This was a continuation of the work from von Neumann and Morgenstein in their 1944 book, which only covered two person zero-sum games, and was restrained by the implications of 'rational' behaviour. \\
\noindent
In 1980, political scientist Robert Axelrod set up a multi-agent tournament for the iterated/repeated prisoner's dilemma. Multiple well-known game theorists from different professions such as psychology, political science, economics, mathematics and more submitted 14 FORTRAN (Formula Translation) programs for the agents to follow as implicit strategies. In this tournament, agents would play against each other for 200 rounds - mutual cooperation would yield 3 points, mutual defection 1 point, single defection 5 points and single cooperation 0 points. The winning strategy was a simple tit-for-tat program which cooperated on the first turn, then repeated the opponents previous move for each subsequent turn. This strategy ended the tournament with an average of 504.5 points of a maximum 1000. \\

\subsection{Prisoner's Dilemma}
The prisoner's dilemma is one of the fundamental games of game theory which shows the payoffs and consequences of two 'players' acting in their own self interests. This summary, cited from britannica.com, is a model version of the prisoner's dilemma: \\

\noindent
"\textit{Two prisoners are accused of a crime. If one confesses and the other does not, the one who confesses will be released immediately and the other will spend 20 years in prison. If neither confesses, each will be held only a few months. If both confess, they will each be jailed 15 years. They cannot communicate with one another. Given that neither prisoner knows whether the other has confessed, it is in the self-interest of each to confess himself. Paradoxically, when each prisoner pursues his self-interest, both end up worse off than they would have been had they acted otherwise.}"

\clearpage

\end{document}